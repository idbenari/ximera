\documentclass{ximera}
\title{Installing locally: Linux}

\begin{document}
\begin{abstract}
Instructions for installing Ximera locally.
\end{abstract}
\maketitle



\section{Install ximeraLaTeX}

\section{For Linux}

Use git to clone the repo with:

\begin{verbatim}
git clone https://github.com/XimeraProject/ximeraLatex.git
\end{verbatim}

Create the directory structure:

\begin{verbatim}
~/texmf/tex/latex/
\end{verbatim}

and move \verb|ximeraLatex| to \verb|~/texmf/tex/latex/|. This will allow all of
your documents to find \verb|ximera.cls|



\section{Install xake}

Xake is Ximera's version of ``make'' and it converts the
\LaTeX\ source into HTML.

\subsection{Installing xake on Arch}

\verb!yaourt -S xake-git! should work assuming you use \texttt{yaourt}.

  
\subsection{Installing xake on Ubuntu}

Download the \verb|.deb| file from here:

\link{https://github.com/XimeraProject/xake/releases}

Then in the directory, do

\begin{verbatim}
sudo apt install ./xake_0.9.2_amd64.deb.deb
\end{verbatim}

where the \verb|0.9.2| will be replaced with the version of the
\verb|.deb| file you have downloaded.

If you want to rely on Ximera's ability to convert TikZ directly to SVGs, then you will need to install a couple other tools,
\begin{verbatim}
sudo apt-get install mupdf-tools
sudo apt-get install pdf2svg
\end{verbatim}

\subsection{Installing xake on Red Hat}

Download the \verb|.rpm| file from here:

\link{https://github.com/XimeraProject/xake/releases}

And install as usual.  You will also want to install \verb|mutool| and \verb|pdf2svg|.

\end{document}
