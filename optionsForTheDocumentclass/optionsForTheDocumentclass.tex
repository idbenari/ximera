\documentclass{ximera}
\title{Options for the documentclass}
\begin{document}
\begin{abstract}
  We describe options for the Ximera documentclass. 
\end{abstract}
\maketitle

There are a number of options for the document class, though their
effects are only seen in the PDF:

%% Copied from dtx
\begin{description}
\item[\texttt{handout}] The default behavior of the class is to display \textbf{all} content. This means that if any questions are asked, all answers are shown. Moreover, some content will only have a meaningful presentation when displayed online. When compiled without any options, this content will be shown too. This option will supress such content and generate a reasonable printiable ``handout.''
\item[\texttt{noauthor}] By default, authors are listed at the bottom of the first page of a document. This option will supress the listing of the authors.
\item[\texttt{nooutcomes}] By default, learning outcomes are listed at the bottom of the first page of a document. This option will supress the listing of the learning outcomes.
\item[\texttt{instructornotes}] This option will turn on (and off) notes written for the instructor.
\item[\texttt{noinstructornotes}] This option will turn off (and on) notes written for the instructor.
\item[\texttt{hints}] When the \texttt{handout} options is used, hints are not shown. This option will make hints visible in handout mode.
\item[\texttt{newpage}] This option will start each problem-like environment (\texttt{exercise}, \texttt{question}, \texttt{problem}, and \texttt{exploration}) start on a new page.
\item[\texttt{numbers}] This option will number the titles of the activity. By default the activities are unnumbered.
\item[\texttt{wordchoicegiven}] This option will replace the choices shown by \texttt{wordChoice} with the correct choice. No indication of the \texttt{wordChoice} environment will be shown.
\end{description}


\end{document}
